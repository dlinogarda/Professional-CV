
\documentclass[icon]{mycvtemplate}

\usepackage{fontspec}
\usepackage{fontawesome5}
\usepackage{tcolorbox}
\tcbuselibrary{skins,breakable}
\usepackage{mdframed}
\usepackage{lipsum}

\begin{document}

%%%%%%%%%%%%%%%%%%%%%%
%% PROFILE SIDE BAR %%
%%%%%%%%%%%%%%%%%%%%%%
% personal info
\profilepic{closeup_lino.jpg}   % path of profile pic
\cvname{Lino Garda Denaro}      % your name
\cvjobtitle{Post-Doct}         % your actual job position

\cvmail{dlinogarda@tutanota.com}
\cvmaill{dlinogarda@ntu.edu.tw}
\cvsite{www.dlinogarda.com} 
\cvgithub{https://github.com/dlinogarda} 
\cvorcid{ORCID: 0000-0001-5367-5392} 
\cvpublons{Web of Science} 
\cvrg{ResearchGate}  
\cvlinkedin{LinkedIn} 

\aboutme{
Lino was born in Batam, Indonesia. He received the B.Eng. degree in Department of Geomatics Engineering from Institute of Technology Sepuluh Nopember (ITS) Surabaya, Indonesia (2011-2015). Afterward, He got a scholarship in Taiwan to continue studying in the geomatics field for his master's degree (2016-2018) and followed by his doctoral degree (2018-2022). He graduated his Doctoral degree program in February 2022 with 6 papers and 1 paper in the submission process.
} % about me section

%%%%%%%%%%%%%%%%%%%%%%%%%%%%%%%%%%%%%%%%%%%%%%%%%%%%%%%%%%%%%%%%%%%%%%%%%%%%%%%%%%%%%
%%%%%% Skill bar section, each skill must have a value between 0 an 6 (float) %%%%%%%
%%%%%%%%%%%%%%%%%%%%%%%%%%%%%%%%%%%%%%%%%%%%%%%%%%%%%%%%%%%%%%%%%%%%%%%%%%%%%%%%%%%%%
\skills{{Programming/ 4},{Remote Sensing/ 4.8},{Deep Learning/ 4.3},{Change Detection Analysis/ 5},{Digital Image Processing/ 4.5},{Reinforcement Learning/ 3},{GAN/ 3.3}}

%%%%%%%%%%%%%%%%%%%%%%%%%%%%%%%%%%%%%%%%%%%%%%%%%%%%%%%%%%%%%%%%%%%%%%%%%%%%%%%%%%%
%%%%%% Skill text section, each skill must have a value between 0 an 6 %%%%%%%%%%%%
%%%%%%%%%%%%%%%%%%%%%%%%%%%%%%%%%%%%%%%%%%%%%%%%%%%%%%%%%%%%%%%%%%%%%%%%%%%%%%%%%%%
%\skillstext{{Reinforcement Learning/ 3},{GAN/ 3.3}}

\makeprofile
%%%%%%%%%%%%%%%%%%%%%%%%%%
%% END PROFILE SIDE BAR %%
%%%%%%%%%%%%%%%%%%%%%%%%%%

%%%%%%%%%%%%%%%%%%
%%%%%% BODY %%%%%%
%%%%%%%%%%%%%%%%%%

%%%%%%%%%%%%%%%%%%%%
%% SIMPLE SECTION %%
%%%%%%%%%%%%%%%%%%%%

{\huge\color{mainblue}Lino Garda Denaro, B.Eng., M.S., Ph.D.}

\begin{flushright}
{\small\color{mainblue}Science of Geomatics and Remote Sensing}
\end{flushright}

\section{interests}
Remote Sensing, Atmospheric Correction, Change Detection Analysis, and Artificial Intelligence

\section{education}

%%%%%%%%%%%%%%%%%%%%%%%%%%%%%%%%%%%%%%%%%%%%%%%%%%%%%%%
%%%%%%%%%%%%% TWENTY LIST ITEMS %%%%%%%%%%%%%%%%%%%%%%%
%% Four arguments: date; title; where; description %%%%
%%%%%%%%%%%%%%%%%%%%%%%%%%%%%%%%%%%%%%%%%%%%%%%%%%%%%%%
\begin{TableTemplate}
  \TableTemplateItem
    {2018 - 2022}
    {Doctor of Philosophy (GPA: 3.9)}
    {Tainan, Taiwan}
    {Department of Geomatics\\ National Cheng Kung University (NCKU)}
  \TableTemplateItem
    {2016 - 2018}
    {Master of Science (GPA: 3.9)}
    {Tainan, Taiwan}
    {Department of Geomatics\\ National Cheng Kung University (NCKU)}
  \TableTemplateItem
    {2011 - 2015}
    {Bachelor of Engineering (GPA: 3.38)}
    {Surabaya, Indonesia}
    {Department of Geomatics Engineering\\ Institute Technology Sepuluh Nopember (ITS)}
  \TableTemplateItem
    {2008 - 2011}
    {High School}
    {Kediri, Indonesia}
    {Natural Science\\ National High School 4 (SMAN-4 Kediri)}
\end{TableTemplate}

%%%%%%%%%%%%%%%%%%%%%%%%%%%%%%%%%%%%%%%%%%%%%%%
%%%%%%%%% TWENTY LIST SHORTITEMS %%%%%%%%%%%%%%
%%% Two arguments: date; title/description %%%%
%%%%%%%%%%%%%%%%%%%%%%%%%%%%%%%%%%%%%%%%%%%%%%%

\section{professional experiences}
\begin{TableTemplate}
  \TableTemplateItem
    {2022 - Now}
    {Post-Doctoral Fellow}
    {National Taiwan University - Taipei, Taiwan}
    {Remote Sensing Laboratory}
  \TableTemplateItem
    {2018 - 2022}
    {Student Research Assistant}
    {Department of Geomatics - NCKU, Taiwan}
    {Geo-Artificial Intelligence Laboratory}
  \TableTemplateItem
    {2016 - 2018}
    {Student Research Assistant}
    {Department of Geomatics - NCKU, Taiwan}
    {Digital Geometry Laboratory}
\TableTemplateItem
    {2015 - 2016}
    {Working in Government Institution}
    {BAPPEKO - Surabaya, Indonesia}
    {Development Planning Agency of Surabaya City}
\TableTemplateItem
    {2014}
    {Internship of Department of Remote Sensing}
    {Jakarta, Indonesia}
    {Lembaga Penerbangan dan Antariksa Nasional (LAPAN)}
\end{TableTemplate}

%%%%%%%%%%%%%%%%%%%%%%%%%%%%%%%%%%%%%%%%%%%%%%%
\section {skills}
\begin{litemize}
  \begin{litem}
  \textbf{Remote Sensing}\\
  Change Detection Analysis, Land Analysis, Ocean Color (Chl-a, Total Suspended Solid), Air Pollution (PM2.5,10), Ground Water Analysis, Hotspot Area, Environmental Analysis
  \end{litem}

  \begin{litem}
  \textbf{Digital Image Processing}\\
  Histogram Processing, Image Filtering (Frequency and Spatial Domains using Fourier Transform), Morphological Image, Restoration and Reconstruction, Image Segmentation, Feature Extraction, and Image Pattern Classification
  \end{litem}

  \begin{litem}
   \textbf{Programming}\\
   MATLAB, Python (Anaconda, PyCharm, Jupyter, Colab), C\#, SQL, R, Java, Java Script
  \end{litem}

 \begin{litem}
   \textbf{Machine Learning}\\
   Deep Learning (Dense Neural Network, Convolutional Neural Network), Water-Net, U-Net, Reinforcement Learning, Transfer Learning
  \end{litem}

 \begin{litem}
   \textbf{Computer Programs}\\
   	Remote Sensing Software (BEAM Visat, SNAP, ENVI, Q-GIS, Arc-Map, etc), Video Editor (Adobe premier Pro), 3D Modelling (PIX 4D Mapper, AutoCAD, Agisoft Photoscan, Australis, Cloud Compare), 2D Design (Corel Draw, Photoshop), Office Applications, LaTeX
 \end{litem}
\end{litemize}

%%%%%%%%%%%%%%%%%%%%%%%%%%%%%%%%%%%%%%%%%%%%%%%

\section {reviewer}
\begin{litemize}
  \begin{litem}
  \textbf{IEEE - Journal of Selected Topics in Applied Earth Observations and Remote Sensing}
  \end{litem}
  \begin{litem}
  \textbf{IEEE - Transaction on Geoscience and Remote Sensing}
  \end{litem}
  \begin{litem}
  \textbf{IEEE - International Journal of Remote Sensing}
  \end{litem}
\end{litemize}


\section {publications}
\begin{TableTemplate}
    \TableTemplateItem
    {Feb. 2020}
    {IEEE Journal of Selected Topics in Applied Earth Observations and Remote Sensing}
    {}
    {Hybrid canonical correlation analysis and regression for radiometric normalization of cross-sensor satellite imagery}
    \TableTemplateItem
    {Jul. 2019}
    {IGARSS 2019- IEEE International Geoscience and Remote Sensing Symposium}
    {}
    {Nonlinear relative radiometric normalization for Landsat 7 and Landsat 8 imagery}
    \TableTemplateItem
    {Mar. 2019}
    {IEEE Geoscience and Remote Sensing Letters}
    {}
    {Pseudoinvariant feature selection using multitemporal MAD for optical satellite images}
    \TableTemplateItem
    {Feb. 2019}
    {The International Archives of Photogrammetry, Remote Sensing and Spatial Information Sciences}
    {}
    {Hybrid Canonical Correlation Analysis and Regression for Radiometric Normalization of Cross-Sensor Satellite Images}
    \TableTemplateItem
    {Nov. 2019}
    {ISPRS Journal of Photogrammetry and Remote Sensing}
    {}
    {Spectral-consistent relative radiometric normalization for multitemporal Landsat 8 imagery}
    \TableTemplateItem
    {Aug. 2018}
    {Journal of Applied Remote Sensing}
    {}
    {Pseudoinvariant feature selection for cross-sensor optical satellite images}
\end{TableTemplate}


\section {conferences}
\begin{twentyshort}
    \twentyitemshort 
    {2021}
    {Asian Conference on Remote Sensing (ACRS), Vietnam}
    \twentyitemshort
    {2019}
    {PhilGeos (GeoAdvences) Paper 2, Philippine}
    \twentyitemshort
    {2019}
    {PhilGeos (GeoAdvences) Paper 1, Philippine}
    \twentyitemshort
    {2019}
    {Asian Conference on Remote Sensing (ACRS), Korea}
    \twentyitemshort
    {2019}
    {International Symposium on Remote Sensing (ISRS), Taiwan}
    \twentyitemshort
    {2018}
    {IEEE International Geoscience and Remote Sensing Symposium (IGARSS), Japan}
    \twentyitemshort
    {2018}
    {International conference of SG37, Taiwan}
    \twentyitemshort
    {2017}
    {International Conference of Indonesia Society for Remote Sensing (ICOIRS), Indonesia}
    \twentyitemshort
    {2016}
    {International Symposium on GNSS (ISGNSS), Taiwan}
    \twentyitemshort
    {2016}
    {International conference of SG36, Taiwan}
\end{twentyshort}

\section {awards \& achievements}
\begin{twentyshort}
    \twentyitemshort 
    {2018 - 2022}
    {Ph.D Distinguished Scholarship - NCKU, Taiwan}
    \twentyitemshort
    {2018 - 2022}
    {Ministry of Science and Technology (MOST) Scholarship, Taiwan}
    \twentyitemshort
    {2019}
    {Invited Speaker of PhilGeos Conference, Philippine}
    \twentyitemshort
    {2017}
    {Best Presenter of International Conferences of Indonesian Society for Remote Sensing (ICOIRS), Indonesia}
    \twentyitemshort
    {2016 - 2018}
    {M.Sc Distinguished Scholarship - NCKU, Taiwan}
    \twentyitemshort
    {2016 - 2018}
    {Ministry of Science and Technology (MOST) Scholarship, Taiwan}
    \twentyitemshort
    {2015}
    {Analyst Surveyor License (Indonesian Surveyors Association)}
    \twentyitemshort
    {2013 - 2015}
    {Van Deventer-Maas Scholarship (VDMS)}
\end{twentyshort}


\section{-----------------------------------------------------------------}

%%%%%%%%%%%%%%%%%%%%%%
%%%%% ENDBODY %%%%%%%%
%%%%%%%%%%%%%%%%%%%%%%

\end{document} 


%%%%%%%%%%%%%%%%%%%%%%%%%%%%%%%%%%%%%%%%%%%%%%%%%%%%%%%%%%%%%%%%%%%%%%%%%%%%
%
% Twenty Seconds Curriculum Vitae in LaTex
% ****************************************
% 
% License: MIT
%
% For further information please visit:
% https://github.com/spagnuolocarmine/TwentySecondsCurriculumVitae-LaTex
% 
%%%%%%%%%%%%%%%%%%%%%%%%%%%%%%%%%%%%%%%%%%%%%%%%%%%%%%%%%%%%%%%%%%%%%%%%%%%%
